\documentclass[dvipsnames,a4paper,11pt]{article}
\usepackage[brazil]{babel}
\usepackage[T1]{fontenc}
\usepackage[utf8x]{inputenc}
\usepackage{graphicx}
\usepackage{hyperref}
\usepackage{amssymb}
\usepackage{color}
\usepackage{amsmath}
\usepackage{listings} 
\usepackage[section]{placeins}

% You can include more LaTeX packages here 

\definecolor{mygreen}{rgb}{0,0.6,0}
\definecolor{mygray}{rgb}{0.5,0.5,0.5}
\definecolor{mymauve}{rgb}{0.58,0,0.82}

\lstset{ %
	backgroundcolor=\color{white},   % choose the background color; you must add \usepackage{color} or \usepackage{xcolor}
	basicstyle=\small,        % the size of the fonts that are used for the code
	breakatwhitespace=false,         % sets if automatic breaks should only happen at whitespace
	breaklines=true,                 % sets automatic line breaking
	captionpos=b,                    % sets the caption-position to bottom
	commentstyle=\color{mygreen},    % comment style
	escapeinside={\%*}{*)},          % if you want to add LaTeX within your code
	extendedchars=true,              % lets you use non-ASCII characters; for 8-bits encodings only, does not work with UTF-8
	keepspaces=true,                 % keeps spaces in text, useful for keeping indentation of code (possibly needs columns=flexible)
	keywordstyle=\color{blue},       % keyword style
	language=Python,                 	   % the language of the code
	numbers=left,                    % where to put the line-numbers; possible values are (none, left, right)
	numbersep=5pt,                   % how far the line-numbers are from the code
	numberstyle=\tiny\color{mygray}, % the style that is used for the line-numbers
	rulecolor=\color{black},         % if not set, the frame-color may be changed on line-breaks within not-black text (e.g. comments (green here))
	showspaces=false,                % show spaces everywhere adding particular underscores; it overrides 'showstringspaces'
	showstringspaces=false,          % underline spaces within strings only
	showtabs=false,                  % show tabs within strings adding particular underscores
	%  stepnumber=2,                    % the step between two line-numbers. If it's 1, each line will be numbered
	stringstyle=\color{mymauve},     % string literal style
	tabsize=2,	                   % sets default tabsize to 2 spaces
	title=\lstname                   % show the filename of files included with \lstinputlisting; also try caption instead of title
}

\usepackage[a4paper,top=2.5cm,bottom=2.5cm,left=3cm,right=3cm,marginparwidth=1.75cm]{geometry}

\begin{document}

%\selectlanguage{english} %%% remove comment delimiter ('%') and select language if required

\author{Bruno da silva}

\fontfamily{phv}\selectfont
\begin{titlepage}
	\begin{center}
		{\scshape\Large Universidade Federal Fluminense -UFF \par}
		\vspace{7cm}
		{\huge\bfseries Animações com Matplotlib \par}
		\vspace{6cm}
		{\itshape Autor Bruno da Silva Machado \par}      
			
		\vspace{6.5cm}    
			
		\vfill
		{\large \today\par}
	\end{center}
\end{titlepage}

\setcounter{secnumdepth}{0}
\section{Introdu\c{c}\~{a}o}
\noindent

Matplotlib \'{e} uma biblioteca de plotagem para o linhagem de programa\c{c}\~{a}o python e uma extens\~{a}o do numpy (numerical python). Ele fornece uma API orientada a objetos incorporada em aplica\c{c}\~{o}es que precis\~{o}a utilizar gr\'{a}ficos. Escolhemos o seu uso por ser uma ferramenta integrada com o python e por sua simplicidade. Nesse trabalho a usaremos para criar animações de modo que possamos ver o comportamento de um pendulo ao longo do tempo. 
\noindent \eject 

\section{Animações com Matplotlib}
\noindent 

As ferramentas de animação se centram em torno da matplotlib.animation.Animation classe base, que fornece uma estrutura em torno da qual a funcionalidade de animação é construída. As principais interfaces são TimedAnimation e FuncAnimation , nas quais podem ser lidas na documentação $(http://matplotlib.sourceforge.net/api/animation_api.html)$.Nesse trabalho vamos usar a FuncAnimation ferramenta, comumente a mais usada nas animações

Primeiro usaremos FuncAnimation para fazer uma animação básica de um pendulo simples sujeito as seguintes hipóteses:

\begin{enumerate}
\item O braço é formado por um fio não flexível que se mantém sempre com o mesmo formato e comprimento.
\item Toda a massa, ${\displaystyle m\,}$, do pêndulo está concentrada na ponta do braço a uma distância constante ${\displaystyle L\,}$ do eixo.
\item Não existem outras forças a atuar no sistema senão a gravidade e a força que mantém o eixo do pêndulo fixo. (O movimento é portanto conservativo).
\item O pêndulo realiza um movimento bidimensional no plano xy.
\end{enumerate}
equação do pendulo simples
\begin{equation}
{d^{2}\theta  \over dt^{2}}+{g \over L}\sin \theta =0.
\end{equation}

e por fim animamos um pendulo forçado que é um pêndulo amortecido o torque de uma força periódica $F = F_0 \sin(\omega_D t)$,onde $\omega_D$ t é a frequência angular da força externa, a segunda lei de Newton para rotação deste movimento torna-se:

\begin{equation}
\ddot{\theta} A \sin(\omega_Dt) − \gamma\dot{\theta} − \omega^2\dot{\theta}\sin(\theta)\
\end{equation}

\paragraph{}
\textit{Código em Python}
\begin{lstlisting}
fig = plt.figure(facecolor='white')
line1,=XxT.plot([],[],'k-',lw=2)
\end{lstlisting}
Aqui criamos uma janela de figura, com um único eixo na e criamos nosso objeto de linha que será modificado na animação. Observe que aqui simplesmente traçamos uma linha vazia.
Em seguida, criaremos as funções que fazem a animação acontecer. init() É a função que será chamada para criar o quadro base sobre o qual a animação ocorre. Aqui, usamos apenas uma função simples que define os dados da linha para nada. É importante que esta função retorne o objeto de linha, pois isso indica ao animador quais objetos no enredo para atualizar após cada quadro:
\begin{lstlisting}
def init():
	line1.set_data([],[])
	line2.set_data([],[])
	line3.set_data([],[])
	line4.set_data([],[])
	return line1,line2,line3,line4,
\end{lstlisting}
A próxima peça é a função de animação. É preciso um único parâmetro, o número do quadro i

\begin{lstlisting}
#funcao animacao
def animate(i):
	a = t[:i]
	b = x[:i]
	c = v[:i]
	d = e[:i]

	
	line1.set_data(a,b)
	line2.set_data(a,c)
	line3.set_data(b,c)
	line4.set_data(a,d)
	return line1,line2,line3,line4,
\end{lstlisting}
Note-se que novamente aqui devolvemos uma tupla dos objetos do da cena que foram modificados. Isso diz ao quadro de animação quais partes do gráfico devem ser animadas.
Finalmente, criamos o objeto de animação:

\begin{lstlisting}	
#cria animacao
anim = animation.FuncAnimation(fig,animate,init_func = init, frames=t.size,interval=0,blit=True,repeat = False)
\end{lstlisting}


\section{Referencia}

\noindent 

N. J. Giordano \& H. Nakanishi, Computational Physics, 2°ed, 2007

3)M. A. Savi, "Din\^amica n\~ao-linear e caos", E-papers, Rio de Janeiro, 2006.


\end{document}

